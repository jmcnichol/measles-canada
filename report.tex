\documentclass[12pt,a4paper]{article}
\usepackage[margin=0.75in]{geometry}
\usepackage{amsmath, amsfonts, amssymb,amsthm,epsfig,epstopdf,titling,url,array}
\usepackage{subfig}
\usepackage{graphicx}
\usepackage{hyperref}
\usepackage{amsbsy}
\usepackage{epsfig}
\usepackage{subfig}
\usepackage{bm}
\usepackage{xspace}
\usepackage{color}
\usepackage{colortbl}
\usepackage{subfloat}
\usepackage{lineno}
\usepackage{authblk}
\usepackage[section]{placeins}
\DeclareMathOperator{\Tr}{Tr}
% \renewcommand{\familydefault}{cmss}
\usepackage{setspace}


\setlength{\parindent}{0.5cm}
\setlength{\parskip}{6pt}

    \newcommand{\beginsupplement}{%
        \setcounter{table}{0}
        \renewcommand{\thetable}{\arabic{table}}%
        \setcounter{figure}{0}
        \renewcommand{\thefigure}{\arabic{figure}}%
        \setcounter{equation}{0}
        \renewcommand{\theequation}{\arabic{equation}}%
     }


\newcommand{\MH}[1]{{\color{red}#1}}
\newcommand{\cc}[1]{{\color{blue}#1}}
\newcommand{\js}[1]{{\color{green}#1}}
\newcommand{\comment}[1]{}

\begin{document}

\title{Measles Outbreaks in Canada: a short modelling study
}


\author[1]{Jennifer McNichol,$^\dag$}
\author[1]{Javad Valizadeh,$^\dag$}
\author[1]{Samara Chaudhury}
\author[1,*]{Caroline Colijn}

\affil[1]{Department of Mathematics, Simon Fraser University, Burnaby, BC, Canada, V5A 1S6}
\affil[*]{Corresponding author: ccolijn@sfu.ca}

\maketitle

\section{Introduction}

Measles is an infectious disease caused by the measles virus. It can cause serious complications, including ear infections and diarrhoea, but also pneumonia, encephalitis and in rarer cases, death. While measles can be serious in any age group, those under 5 years old, over 20, and those who are immune-compromised are most at risk. Measles is highly contagious, with a basic reproduction number of 12-18.
In this brief report we use a simple stochastic simulation model to explore how large measles outbreaks in Canada could be, based on past outbreak sizes in similar populations, parameters for the transmissibility of measles and the course of infection, and Canadian heterogenous vaccination rates.

\section{Methods}
Vaccination in Canada
Canada has high levels of measles vaccination overall, with Statistics Canada reporting 90\% or higher coverage of recommended measles vaccines in the 2017-2021 period (https://www150.statcan.gc.ca/t1/tbl1/en/tv.action?pid=1310087001). However, there is considerable variability between regions and communities, with some areas, schools or geographies vulnerable to measles outbreaks due to considerably lower vaccination rates. Table 1 lists provincial and territorial data sources and whether vaccination is required or recommended for school-age children. We show some of these data in Figures 1 and 2.


% INSERT TABLE 1  from the google doc
\begin{table}
  \centering

  \caption[Vaccination requirements by province]{Vaccination requirements in schools in Canadian regions, with data and/or reports of vaccination rates}
  \label{tab:vaxrequire}
  \begin{tabular}{llp{5cm}}
Province & Required to start school? & Coverage rates \& provincial reports                                                                                                                                                \\ \hline
NFLD     & recommended               & A recent news interview states 95\% of the pop https://www.cbc.ca/player/play/2308385347572\#:$\sim$:text=The\%20key\%20to\%20protection\%20is,was\%20a\%20traveller\%20in\%202017. \\ \hline
PEI      & recommended               & Rates prior to grade 1 entry for 2022-2023: https://www.princeedwardisland.ca/en/publication/childhood-immunization-rates-year-20222023-school-age                                  \\ \hline
NS       & recommended               & Data not found                                                                                                                                                                      \\ \hline
NB       & Required                  & NA                                                                                                                                                                                  \\ \hline
QC       & recommended               & Data not found, but some news reports suggest some schools as low as 75\%                                                                                                                \\ \hline
ON       & Required                  & NA                                                                                                                                                                                  \\ \hline
MB       & Required                  & NA                                                                                                                                                                                  \\ \hline
SK       & recommended               & https://www.saskatchewan.ca/residents/health/accessing-health-care-services/immunization-services/immunization-rates-in-saskatchewan                                                \\ \hline
AB       & recommended               & http://www.ahw.gov.ab.ca/IHDA\_Retrieval/selectSubCategory.do                                                                                                                       \\ \hline
BC       & recommended               & http://www.bccdc.ca/health-professionals/data-reports/immunizations                                                                                                                 \\ \hline
Yukon    & recommended               & Not found                                                                                                                                                                           \\ \hline
Nunavut  & recommended               & Not found                                                                                                                                                                           \\ \hline
NWT      & recommended               & Not found
\end{tabular}
\end{table}


\begin{table}[]
\caption{}
\label{tab:my-table}
\begin{tabular}{lp{5cm}p{5cm}p{5cm}}
\hline
Parameter &
  Value &
  Description &
  Reference or rationale \\ \hline
$R_0$ &
  15 &
  Basic reproduction number. &
  \begin{tabular}[c]{@{}l@{}}Reported range 12-18.\\ {[}REFS{]}\end{tabular} \\ \hline
$c$ &
  0.3 &
  “Exposed” individuals are infectious before onset of rash &
  4 days infectiousness prior to rash onset \\ \hline
$v$ &
  0.005 per day (strong), 0 (weaker) &
  Supplementary vaccination of susceptibles. Not all jurisdictions strongly encourage additional vaccination in the general population as an outbreak response. &
  \begin{tabular}[c]{@{}l@{}}British Columbia vaccination in 2019 outbreak (rough estimate)\\ {[}REF{]}\end{tabular} \\ \hline
$q_s$ &
  0.03 per day &
  Isolation of susceptible individuals following contact tracing, exposure notifications &
  See caption \\ \hline
  $q_{spep}$ &
  0.05 per day (strong); 0.04 (weak) &
  Post-exposure prophylaxis (PEP) plus isolation rate if PEP declined &
  See caption \\ \hline
$k$ &
  1/10 per day &
  Progression to rash onset &
  Course of infection (WHO) Rash onset at 7-18 days \\ \hline
$q_i$ &
  0.15 per day (strong); 0.1 per day (weaker) &
  Identification and isolation of symptomatic individuals &
  See caption \\ \hline
$\gamma $&
  1/4 per day &
  Recovery from symptoms and infectiousness &
  Infectious until 4 days after rash onset (WHO) \\ \hline
\end{tabular}
\end{table}

\end{document}
